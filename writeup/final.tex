\documentclass{article}

% if you need to pass options to natbib, use, e.g.:
% \PassOptionsToPackage{numbers, compress}{natbib}
% before loading nips_2017
%
% to avoid loading the natbib package, add option nonatbib:
% \usepackage[nonatbib]{nips_2017}

\usepackage[final]{nips_2017}

% to compile a camera-ready version, add the [final] option, e.g.:
% \usepackage[final]{nips_2017}

\usepackage[utf8]{inputenc} % allow utf-8 input
\usepackage[T1]{fontenc}    % use 8-bit T1 fonts
\usepackage{hyperref}       % hyperlinks
\usepackage{url}            % simple URL typesetting
\usepackage{booktabs}       % professional-quality tables
\usepackage{amsfonts}       % blackboard math symbols
\usepackage{nicefrac}       % compact symbols for 1/2, etc.
\usepackage{microtype}      % microtypography

\title{Deep Learning and Determining Duplicate Questions}

% The \author macro works with any number of authors. There are two
% commands used to separate the names and addresses of multiple
% authors: \And and \AND.
%
% Using \And between authors leaves it to LaTeX to determine where to
% break the lines. Using \AND forces a line break at that point. So,
% if LaTeX puts 3 of 4 authors names on the first line, and the last
% on the second line, try using \AND instead of \And before the third
% author name.

\author{
  Jacob L. Fenger\\
  Oregon State University\\
  \texttt{fengerj@oregonstate.edu} \\
  %% examples of more authors
  \And
  Spike Madden \\
  Oregon State University\\
  \texttt{maddens@oregonstate.edu} \\
  \AND
  Chongxien Chen \\
  Oregon State University\\
  \texttt{chencho@oregonstate.edu} \\
}

\begin{document}
% \nipsfinalcopy is no longer used

\maketitle

\begin{abstract}
  This document contains the approach Team Jacob utilized to train a classifier
  for detecting whether questions are duplicates. We utilized Word2Vec embeddings
  for vectorizing the questions as well as Keras with a Tensorflow backend
  for training.
\end{abstract}

\section{Introduction}

Quora Question Pairs is a Kaggle competition requiring challengers to use
advanced techniques to determine if pairs of questions are duplicates. You
can find the challenge at:
\begin{center}
\textit{https://www.kaggle.com/c/quora-question-pairs}
\end{center}

Challengers will need to utilize natural language processing as well as
deep learning in order to create a good classification. Our team had no
previous experience with natural language processing and during our
implementation of the challenge, we learned a lot. There are many aspects
to consider when converting sentences to number vectors. This process is
otherwise known as vectorization.

Our first approach got a score of  while our second approach got a score
of. Explanations of these approaches are outlined in the following
sections.

\section{Approach \#1}

The first approach was influenced by a user named Lystdo who posted a kernel
for the challenge. Our approach for the challenge was heavily influenced by
his mainly because we had no experience with the natural language processing
side of things.

\subsection{Text Pre-Processing}

To ensure accurate results, we pre-processed the text. Initially, this
just involved removing all punctuation as well as making everything lowercase.
This is necessary because many questions, especially those in the test set,
containing words with weird capitalisation or spelling. Our knowledge
regarding natural language processing is very limited, but we had confidence
that punctuation in the questions provide no additional meaning.

Later on in the implementation of this challenge, we realized that some
questions involved contractions, slang terms, or spelling mistakes. We tried
our best to handle those by doing replacements with commonly known
contractions/slang terms, but some still may have gotten through due to the
large data set. The testing set was generated by Quora and many of the
questions had spelling mistakes. A user called 'Currie32' had an interesting
discussion post regarding the cleaning of text for the challenge.

\subsection{Feature Extraction}

After reading all of the data and pre-processing the text, a necessary step is
to analyze the text and generate numeric feature vectors because the majority
of machine learning algorithms require them. This process is otherwise known
as vectorizaton.

We first tokenized the questions and then padded the sequences so they were
all the same size. Using a pre-trained word2vec model is very helpful in the
vectorization of things. There are several pre-trained models in existence.
Word2Vec by Google and GloVe by Stanford are just two examples. We decided
to use Word2Vec for our implementation. There is a neat package that helps
with the interface to Word2Vec called Genism which is what we used as well.

\subsection{Training}

We utilized Keras and Tensorflow to build a long short term model (LSTM)
for the classification. This consisted of an embedding layer involving an
embedding matrix generated from the Word2Vec vocab. Other layers included
dense and dropout layers with batch normalization in-between. Batch
normalization helps with the normalization of data between layers (In short).

\subsection{Results}

The score we got with this approach was about %INSERT SCORE.
After updating the text pre-processing with contraction expansion,
we got reduced the log-loss by several points.

\section{Approach \#2}
The second approach was influenced by a user named ‘anokas’ who posted an approach (Kernel) for the challenge. We decided to base our approach off of his and to see if we could get a higher accuracy by making necessary changes.

\subsection{Analyse data}

It is important to understand how our data looks like before we can train a good model. We first analyze how many samples there are in the train and test. Then a couple other things that can help us understanding the data. Are there many typos in the question? How many words are usually in a question? What are the words that appear the most often? How about the punctuations in the question?

\subsection{Feature Extraction}

TF-IDF approach
Using wordcloud, we find that many words that appear very often doesn’t have significant impacts on whether these questions are duplicate or not. For example, from the picture we see that “good” appears extremely often in the questions. But we can’t really determine how similar two questions are based on the word “good”. It could even be misleading if there is a “not” precedent to “good”. 
A more valuable feature to look at will be TF-IDF (term-frequency-inverse-document-frequency). If a word that doesn’t appear very often in the database appears in the two questions we are comparing, then it is reasonable that we weigh this word more in the prediction of whether these two questions are duplicate. 

\subsection{Training}

We split our train.csv into training data and validation data. Using XGBoost, we generate a prediction csv for questions in test.csv. We get a score of 0.35372 on Kaggle.

\section{Comparison of Results}

\section{First Place Solution}

A member of the DL guys, who placed 1st in the competition, provided a summary
of their approach to the Quora Question Pairs. They had a log loss score of
0.11580.

\subsection{Features}

Embedding features include word embeddings, sentence embeddings and encoded
question pairs. As discussed in a previous approach, Word2Vec was used for the
word embeddings. Doc2Vec and Sent2Vec were used for sentence embeddings but
were found to not be as informative as the word embeddings from Word2Vec. Classical text mining methods were used by tokenizing the strings and looking
for similarity measures on bag of character n-grams, question length, capital
letters and punctuation, and certain keywords like are, can and how. Structural
features were extracted from graphs built from the edges between pairs of
questions from the training and testing data sets.

\subsection{Models}

There were two architectures for the neural networks that were used: a siamese
LSTM network with GloVe embeddings and a decomposable attention network with
FastText embedding. The best single models were found to be these neural
networks trained on both text sequences and the text mining and structural
features.

\subsection{Ensemble}

They used stacking with 4 layers to combine their models. The first layer,
consisting of about 300 models, includes various neutral networks, algorithms
like XGB and LGBM as well as Scikit classification algorithms. The second layer
consists of 150 models and includes the input features and hidden layers of the
best L1 pure text EISM - Evolutionary Support Vector Machine Inference Model.
The third layer consists of 2 linear models and the fourth is for blending.

\section{Conclusion}

\appendix
\section*{Individual Contributions}

\end{document}
